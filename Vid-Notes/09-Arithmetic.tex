\section{Arithmetic}

\subsection{Elementary Arithmetic}

\subsubsection*{Addition}

Pseudocode:
\begin{verbatim}
def add(x, y, z):
  carry = 0
  for i in range(0, n):
    carry, z[i] = add_digits(x[i], y[i], carry)
  z[n] = carry
\end{verbatim}

Adding two digits and a carry-in is an \textbf{atomic add} (\verb|add_digits| above).\\
The time for an atomic add is $\alpha$, so the time for the algorithm above is $\alpha n$, where $n$ is the number of digits. You can't do better than linear time.

\subsubsection*{Multiplication}

Multiplying two digits is an \textbf{atomic multiply}.\\
The time for an atomic multiply is $\mu$, so the time for the standard multiplication algorithm is $n^2\mu + 2n(n-1)\alpha$ (ignoring carries)\\
\emph{Can} do better than quadratic time with Karatsuba's algorithm

\subsection{Recursive multiplication}

\subsubsection*{Two-digit multiplication}

Multiplying two 2-digit numbers: $\underline{\mathrm{ab}} \times \underline{\mathrm{cd}} = (10a + b)(10c + d) = 10^2ac + 10(ad + bc) + bd$\\
We can ``add'' $10^2ac$ and $bd$ by simply concatenating the result of $ac$ and the result of $bd$ together.\\
Then we add $10(ad + bc)$ to that in the normal way.

\emph{Every} multiplication is a two-digit multiplication with a large enough base.

Recurrence for the time to multiply:
\[M(1) = \mu, M(n) = 4M\paren{\frac n 2} + A(n) + A(n) = 4M\paren{\frac n 2} + 2\alpha n\]
Un-recurrence-ified: $M(n) = n^2\mu + 2\alpha n(n-1)$ (same as standard algorithm)

\subsection{Karatsuba's algorithm}

Observe that we don't need $ad$ and $bc$ separately, we only need $ad + bc$. $ad + bc = (a + b)(c + d) - (ac + bd)$\\
Now product is $10^2ac + 10((a + b)(c + d) - (ac + bd)) + bd$\\
So we only multiply 3 times ($ac$, $bd$, and $(a + b)(c + d)$)

Recurrence for time to multiply using Karatsuba's:
\[M(1) = \mu, M(n) = 3M\paren{\frac n 2} + 2A\paren{\frac n 2} + 2A(n) + A(n) = 3M\paren{\frac n 2} + 4\alpha n\]

Un-recurrence-ified: $M(n) = (\mu + 8\alpha)n^{\lg 3} - 8\alpha n$

This has a larger constant factor but smaller exponent than standard algorithm, so better for large $n$ (big numbers).

In our examples, an atomic value was base 10, but in real life, it's $2^{64}$.

Better algorithms:
\begin{itemize}
    \item Toom-Cook: Generalizes Karatsuba's so you can get $\Theta(n^c)$ for $c$ arbitrarily close to 1. Trade-offs not completely understood.
    \item Harvey-van der Hoeven: $\Theta(n \log n)$. Only useful for ludicrously large numbers.
\end{itemize}
