\section{Quicksort}

The algorithm:
\begin{enumerate}
    \item Pick out some item, call it the \textbf{pivot}.\\
    Note: Kruskal wants to use the last element as the pivot.
    \item The \hyperref[sec:partition-step]{\textbf{partition}} step.
    \item Repeat on the left side and the right side.
\end{enumerate}

\subsection*{The partition step}\label{sec:partition-step}

Put all smaller items left, all larger items right, and put the pivot in between.
\begin{itemize}
    \item Process list from left to right. Group small values on the left, then large values, and unprocessed values will be at the end.
    \item $i$ keeps track of where the small values end, $j$ keeps track of where the large values end. $p$ is the start of the subarray, $r$ is the end of the subarray.
    \item At the start, $i = p-1$, $j$ really starts at $p$
    \item When processing $A_{j+1}$:
    \begin{itemize}
        \item If it's larger than the pivot, then it's already in the right place, so just increment $j$.
        \item If it's smaller than the pivot, exchange $A_{i+1}$ and $A_{j+1}$ so that it's now at the end of the small values group.\\
        Then increment both $i$ and $j$.
    \end{itemize}
\end{itemize}

Pseudocode:
\begin{verbatim}
function partition(A, p, r)
  X = A[r]
  i = p - 1
  for j in p to r - 1:
    if A[j] <= X:
      i += 1
      swap(A[i], A[j])
  swap(A[i + 1], A[r])
  return i + 1
\end{verbatim}

\subsection*{Best and Worst Case Comparison}

$n-1$ comparisons in one partition step (compare every element to pivot)\\
Total \# of comparisons in worst case (pivot on smallest/largest element): $\displaystyle \frac{n(n-1)}{2}$\\
Total \# of comparisons in best case (pivot on median): $\displaystyle \sim n\lg n$

\subsection*{Average Case Comparisons Approximation}

Assume average case happens when pivot is the $n/4$th value (halfway between smallest value and median, or halfway between median and largest value)

Recurrence for number of comparisons:
\[S(0) = \begin{cases}
    0 & \text{if } n = 0, 1 \\
    S\paren{\frac n 4} + S\paren{\frac{3n}{4}} + (n - 1) & \text{otherwise}
\end{cases}\]

Total \# of comparisons: $\sim n\lg n$

\subsection*{Exact Average Case Comparisons}

Let $S(n)$ be the average number of comparisons\\
Assume the first partition pivots on the $q$th smallest value\\
Then the average number of comparisons to sort will be $(n-1) + S(q-1) + S(n-q)$

Recurrence for average number of comparisons:
\[S(0) = \begin{cases}
    0 & \text{if } n = 0, 1 \\
    \displaystyle (n-1) + \sum_{q=1}^n \frac 1 n \paren{S(q-1) + S(n-q)} & \text{otherwise}
\end{cases}\]

Using constructive induction, we get \fbox{$\sim 2n\ln n$}.

\subsection*{Choosing Pivots}

Risky to pivot on last element. Better choices are
\begin{itemize}
    \item Middle element
    \item Median of first, middle, and last elements
    \item Random element
    \item Median of 3/5/7/whatever random elements
\end{itemize}

\subsection*{Comments on Quicksort Being In-Place}

Quicksort needs $O(\log n)$ space for the call stack, so not constant memory. Best case is if pivot is smallest element, when call stack is empty.

\textbf{Definition:} An algorithm is in-place if it uses $O(1)$ extra variables and (on average) $O(\log n)$ extra index variables.

\textbf{Definition:} Since index variables use $O(\log n)$ bits, we can also say an algorithm is in-place if it uses $O(1)$ extra variables and (on average) $O((\log n)^2)$ extra bits.

\underline{Not a fundamental definition}. Just one Kruskal uses?
