\section{NP-Completeness}

Classes:
\begin{itemize}
    \item \textbf{P}: $O(n^k)$ with constant $k$ (decision problems solvable in polynomial time)
    \item \textbf{E}: $O(b^n)$ with constant $b$ (exponential time)
    \item \textbf{EXP/EXPTIME}: $O(e^{n^k})$ with constant $k$ (\emph{also} exponential time)
    \item \textbf{NP}: Problems where YES answers are verifiable in polynomial time (can use a ``certificate'')
    \item \textbf{NP-hard}: Problems that are at least as hard as the hardest problems in NP
\end{itemize}

P is a subset of NP (no one knows if the reverse is true too)

NP is further partitioned into:
\begin{itemize}
    \item \textbf{P}: The easy problems
    \item \textbf{NP-complete}: The hard problems (NP-complete is the intersection of NP and NP-hard)
    \item \textbf{NPI} (NP Intermediate): The intermediate problems
\end{itemize}

Recording 20.3 has an example of reduction (reducing 2-component coloring to 1-component coloring).

Recording 20.4 shows that Formula SAT and Circuit SAT are equivalent.

\textbf{Proofs of NP-Completeness (Rec 20.5)}

In order to show that some problem $A$ is NP-Complete:
\begin{itemize}
    \item Show that $A$ is in NP
    \item For some NP-complete problem $B$, show that $B \leq_{\mathrm P} A$
\end{itemize}

\subsection*{The Structure of NP (Rec 20.6)}

If a problem $A$ is in P, then its complement $\Bar{A}$ is also in P.

Graph isomorphism is in NP but not known if it's in Co-NP or not (probably in Co-NP though)

P $\subseteq$ NP $\subseteq$ PSPACE $\subseteq$ EXPTIME $\subseteq$ EXPSPACE

\textbf{Undecidable problems}: Can't even solve (e.g. halting problem)

Recording 20.7 has equivalence of decision and optimization for coloring problem


