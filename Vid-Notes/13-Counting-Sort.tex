\section{Counting Sort}
\label{sec:Counting-Sort}

\subsection*{Naive version}

Input: Array \verb|A| with $n$ values, where each value is an integer between 0 and $k-1$, inclusive.\\
Output: Array \verb|A| sorted into array \verb|B|\\
Overview:
\begin{enumerate}
    \item Count how many times each value in \verb|A| occurs
    \item Put the counts into an array \verb|C| so that \verb|C[i]| is the number of times value \verb|i| occurs.
    \item Iterate through \verb|C| putting value \verb|i| into the next \verb|C[i]| locations of \verb|B|.
\end{enumerate}

Total time: $\Theta(n + k)$

But this will only repeat the same element multiple times. It won't work if we want to be able to sort data by a certain property.

\subsection*{Better version}

\begin{enumerate}
    \item Count how many times each value in $A$ occurs
    \item Put the counts into an array \verb|C| so that \verb|C[i]| is the number of times value \verb|i| occurs.
    \item Form the partial sums of the values in \verb|C|
    \item Iterate backwards through \verb|A| putting value \verb|A[i]| into the proper location of \verb|B| using \verb|C| to determine the index.\\
    When you place a value \verb|i| from \verb|A| into \verb|B| at index \verb|C[i]|, decrement \verb|C[i]|.
\end{enumerate}

Pseudocode:
\begin{verbatim}
# Initialize C
for i in range(0, k) do C[i] = 0
# Put counts into C
for i in range(0, n) do C[A[i]] += 1
# Form partial sums
for i in range(1, k) do C[i] += C[i - 1]
# Final step
for j in range(n, 1) do
  B[C[A[j]]] = A[j]
  C[A[j]] -= 1
\end{verbatim}

Total time is again $\Theta(n + k)$
