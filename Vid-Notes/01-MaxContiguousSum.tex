\section{Maximum Contiguous Sum}

\textbf{Definition:} The maximum possible sum gotten by summing a subarray of some array.\\
e.g. in [3, 2, -4, -5, 6, 1, -3, 7, -8, 2], the maximum contiguous sum is 11 (gotten by summing the subarray [6, 1, -3, 7])

The empty list counts as a subarray, so if all the elements are negative, the maximum contiguous sum would be 0 (sum of [])

\subsection*{Brute force solution (cubic)}

The following brute solution has 3 nested loops that iterate roughly n items, so its time complexity is $\Theta(n^3)$:

\begin{verbatim}
M = 0
for i in 1 to n do
  for j in i to n do
    S = 0
    for k in i to j do
      S += A[k]
    M = max(M, S)
\end{verbatim}

We can count more accurately how many times the innermost loop gets executed with summations:
\[\sum_{i=1}^n \sum_{j=i}^n \sum_{k=i}^j 1 = \frac{n(n+1)(n+2)}{6}\]

This confirms that it is cubic.

\subsection*{Smarter solution (quadratic)}

\begin{verbatim}
M = 0
for i in 1 to n do
  S = 0
  for j in 1 to n do
    S += A[j]
    M = max(M, S)
\end{verbatim}

We can again count how many times the inner most loop executes using a summation:

\[\sum_{i=1}^n \sum_{j=i}^n 1 = \frac{n(n+1)}{2}\]

This confirms that it is quadratic.

\subsection*{Dynamic programming solution (linear)}

\begin{verbatim}
M = 0
# The maximum sum that ended on the previous element
S = 0
for i in 1 to n do
  # Max with 0 in case it goes negative (empty subarray always allowed)
  S = max(S + A[i], 0)
  M = max(M, S)
\end{verbatim}

Loop runs $\displaystyle \sum_{i=1}^n 1 = n$ times, so $\Theta(n)$
