\section{Spanning Trees}

\textbf{Tree:} A connected, acyclic graph.

\textbf{Spanning tree:} A subset of the edges of the graph that includes all the vertices but has no cycles. We'll be talking only about undirected graphs.

\textbf{Minimum spanning tree:} A spanning tree where the sum of the edge weights is minimized (if you don't have edge weights, this means you need to have as few edges as possible).

Kruskal's algorithm and Prim's algorithm can be used to find an MST. Both are \textbf{greedy} (at each step, do whatever is locally optimal).

\subsection{Kruskal's algorithm}

Greedy algorithm that builds up the MST edge by edge.

\begin{verbatim}
edges <- sort edges by edge weight
T <- empty tree # our result
for edge in edges:
  if adding edge to T doesn't result in cycle:
    add edge to T
return T
\end{verbatim}

This is $\Theta(E \log E)$ where $E$ is number of edges.\\

\subsection{Prim's algorithm}
\label{sec:prims-algorithm}

Greedy algorithm that grows the MST starting at an arbitrary vertex and keeps track of distance from starting vertex. Similar to \hyperref[sec:dijkstras]{Djikstra's algorithm}, except we add the vertex closest to the current tree rather than the vertex closest to the source vertex.

\begin{enumerate}
    \item Start with any vertex $v$.
    \item Initialize array of distances for each vertex (initially $\infty$ for all of them).
    \item Repeat the following until all vertices are in your tree:
        \begin{enumerate}
            \item Of the edges that connect $v$ to another vertex not in the MST, pick the one that's closest to the tree.
            \item Add that edge to the MST.
            \item Update $v$ to a new vertex (in the MST) whose distance from the starting vertex is minimized.
            \item Update the distance and predecessor for the new $v$ (the predecessor is the old $v$).
        \end{enumerate}
\end{enumerate}

With a normal list that has 1s to represent unvisited vertices and 0s to represent visited vertices and an adjacency matrix to represent the graph itself, this is $\Theta(V^2)$ (better for dense graphs).

Alternatively, can use min heap acting as priority queue to store unvisited vertices, with runtime of $\Theta(E \log V)$ (better for sparse graphs).

See video for pseudocode of the min heap version.
